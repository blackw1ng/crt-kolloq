\section{Modellierung idealer Reaktoren}
\subsection{Ideales Str�mungsrohr (PFTR)}
W�rmebilanz f�r das ideale Str�mungsrohr:
\[ \fbox{$\underbrace{\frac{dT}{dz}}_{\mbox{Konvektion}}=\frac{1}{u\rho c_p}\left(\underbrace{\sum_{j}\left(-\Delta H_j\right)r_j}_{\mbox{W�rmequelle/-senke}}-\underbrace{k_W\frac{A}{V}\left(T-T_K\right)}_{\mbox{W�rmetransport}}$} \]
Beim adiabaten Str�mungsrohr entf�llt der W�rmetransport durch die Wand (einzige W�rmever�nderung durch Reaktion).
F�r eine einzelne Reaktion folgt durch Verkn�pfung mit dem Umsatz und Einf�hrung der adiabaten Temperaturerh�hung $\Delta T_{ad}$
\[ \fbox{$dT=\underbrace{\frac{c_[1,0}\left(-\Delta H_R\right)}{\rho_0\cdot \bar c_p}}_{\Delta T_{ad}}dX=\Delta T_{ad}dX$} \]
Adiabate Trajektorie:\\
\[ X=\frac{T-T_0}{\Delta T_{ad}} \]
\subsection{Verkn�pfung von Masse- und W�rmebilanz im nichtisothermen STR}
Bedingung f�r station�ren Betrieb:
\[ \underbrace{\dot q_{gen}}_{\mbox{W�rmeproduktion}}=\underbrace{\dot q_{rem}}_{\mbox{W�rmeabfuhr}} \]
%hier muss Grafik Seite 15 rein
CPH: curve of production of heat $\rightarrow$ nichtlinear wegen {\sc Arrhenius}\\
RHL: removal of heat line $\rightarrow$ lineare Funktion der Temperatur\\
$\vartheta=\frac{T}{T_0}$: Temperaturerh�hung im Reaktor (Bulk-/Eingangstemperatur)\\
Drei Betriebsweisen:
\begin{itemize}
\item[a] {\it gel�schtes System}\\
Temperaturver�nderung verursacht von selbst einen R�ckgang zu einem einzigen stabilen Betriebspunkt.
\item[b] {\it gez�ndetes System}\\
Wie a) nur auf hohem Temperaturniveau
\item[c] {\it instabiles System}\\
Kleine Temperaturabweichung verursacht Z�nden oder Erl�schen $\rightarrow$ stabiler Betriebspunkt stellt sich ein.
\end{itemize}
Instabiler Betriebspunkt wenn:
\[ \fbox{$\frac{d \dot q_{rem}}{d\vartheta}<\frac{d \dot q_{gen}}{d\vartheta}$} \]
Analoge Darstellung ist im $X-T$-Diagramm m�glich: hier sind RHL's parallel. Ums�tze im instabiler Bereich k�nnen nicht erreciht werden.


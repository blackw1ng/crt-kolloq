\section{Makrokinetik}
\begin{itemize}
\item eingeschr�nkte Transportgeschwindigkeit im Pellet $\rightarrow$ $c\neq T_{bulk}$ 
\item Temperaturgradient im Pellet $\rightarrow$ $T\neq T_{bulk}$
\end{itemize}
$\rightarrow$ mittlere Reaktionsgeschwindigkeit\\
Wichtig: Nur die Parameter in der Bulk-Phase k�nnen gemessen werden.\\
Einfachster Fall: keine Filmdiffusion
\[ \underbrace{\left(c_i,T\right)_b}_{\mbox{bulk-Bedingungen}}=\underbrace{\left(c_i,T\right)_s}_{\mbox{Oberfl�chenbedingungen}} \]
Faustregel: Hauptwiderstand f�r den Massentransport $\rightarrow$ im Pellet - Hauptwiderstand f�r den W�rmetransport $\rightarrow$ im Film\\

Wirkungsgrad:
\[ \eta = \frac{\mbox{mittlere Reaktionsgeschwindigkeit im Pellet}}{\mbox{Reaktionsgeschwindigkeit bei Oberfl�chenbedingungen}} = \frac{r_{e,effektiv}}{r_{s,Oberfl�che}} \]

\subsection{Porendiffusion im isothermen Katalysatorkorn}
Annahmen:
\begin{itemize}
\item keine Filmdiffusion (bulk=surface)
\item isotherm
\item gleichm��ige Porenstruktur $D_e=const.$
\item $A_1 \rightarrow$ Produkte ($c_1=c$)
\item kugelf�rmiges Katalysatorpartikel
\item Katalysatorpartikel ist pseudo-homogenes System
\end{itemize}
1. Fick'sches Gesetz: \[ \fbox{$j=D_e\frac{dc}{dx}$} \]
effektiver Diffusionskoeffizient: \[ \fbox{$D_e=\frac{D\cdot \epsilon}{\tau}$} \]
$\tau$: Tortuisit�t\\
$\epsilon$: Porosit�t\\
grobe N�herung: $D_e \approx \frac{1}{10}D$
\[ \mbox{Diff.-fluss}_{x'+dx}-\mbox{Diff.-fluss}_{x'}=\mbox{Verbrauch durch Reaktion im Volumenelement} \]
\[ j_{x'+dx}-j_{x'}=\underbrace{\Gamma}_{\mbox{Quelle/Senke}} \]


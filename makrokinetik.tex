\section{Makrokinetik}
\begin{itemize}
\item eingeschr�nkte Transportgeschwindigkeit im Pellet $\rightarrow$ $c\neq c_{bulk}$ 
\item Temperaturgradient im Pellet $\rightarrow$ $T\neq T_{bulk}$
\end{itemize}
$\rightarrow$ mittlere Reaktionsgeschwindigkeit\\
Wichtig: Nur die Parameter in der Bulk-Phase k�nnen gemessen werden.\\
Einfachster Fall: keine Filmdiffusion
\[ \underbrace{\left(c_i,T\right)_b}_{\mbox{bulk-Bedingungen}}=\underbrace{\left(c_i,T\right)_s}_{\mbox{Oberfl�chenbedingungen}} \]
Faustregel:\\
Hauptwiderstand f�r den Massentransport $\rightarrow$ im Pellet\\
Hauptwiderstand f�r den W�rmetransport $\rightarrow$ im Film\\

Wirkungsgrad:
\[ \eta = \frac{\mbox{mittlere Reaktionsgeschw. im Pellet}}{\mbox{Reaktionsgeschw. bei Oberfl�chenbedingungen}} = \frac{r_{e,effektiv}}{r_{s,Oberfl.}} \]

\subsection{Porendiffusion im isothermen Katalysatorkorn}
Annahmen:
\begin{itemize}
\item keine Filmdiffusion (bulk=surface)
\item isotherm
\item gleichm��ige Porenstruktur $D_e=const.$
\item $A_1 \rightarrow$ Produkte ($c_1=c$)
\item kugelf�rmiges Katalysatorpartikel
\item Katalysatorpartikel ist pseudo-homogenes System
\end{itemize}
1. {\sc Fick}'sches Gesetz: \[ \fbox{$ \displaystyle j=D_e\frac{dc}{dx}$} \]
effektiver Diffusionskoeffizient: \[ \fbox{$ \displaystyle D_e=\frac{D\cdot \varepsilon}{\tau}$} \]
$\tau$: Tortuisit�t ("Labyrinth-Faktor")\\
$\varepsilon_p$: Porosit�t\\
{\sc Bruggemann}-Gleichung: $\tau=\frac{1}{\sqrt{\varepsilon_p}}$\\
grobe N�herung: $D_e \approx \frac{1}{10}D$\\
Massenbilanz:\\
\[ \mbox{(Diff.-fluss)}_{x'+dx}-\mbox{(Diff.-fluss)}_{x'}=\mbox{Verbr. d. Rkt. im Vol.-element} \]
\[ j_{x'+dx}-j_{x'}=\underbrace{\Gamma}_{\mbox{Quelle/Senke}} \]
Nach Vereinfachung:
\[ \fbox{$ \displaystyle \frac{d^2c}{dx'^2}+\frac{2}{x'}\frac{dc}{dx'}=\frac{kc^n}{D_e}$} \]
Randbedingungen:\\
$x'=R \qquad \rightarrow \qquad c=c_s$\\
$x'=0 \qquad \rightarrow \qquad \frac{dc}{dx'}=0$\\
{\sc Thiele}-Modul:\\
\[ \fbox{$ \displaystyle \phi=R\cdot \sqrt{\frac{k\cdot c_s^{n-10}}{D_e}}=\sqrt{DaII}=\frac{\mbox{Reaktion}}{\mbox{Diffusion}}$} \]
$DaII$: {\sc Damk�hler}-Zahl 2. Ordnung\\
$\phi$ klein: niedrige Reaktionsgeschwindigkeit; Reaktion limitiert Gesamtgeschwindigkeit\\
$\phi$ gro�: hohe Reaktionsgeschwindigkeit; Diffusion limitiert Gesamtgeschwindigkeit\\
F�r andere Geometrien $\rightarrow$ modifiziertes Thiele-Modul:
\[ \phi_P=\frac{V_P}{O_P}\cdot \sqrt{\frac{k\cdot c_s^{n-10}}{D_e}} \]
$V_P$: Volumen des Pellets\\
$O_P$: �u�ere Oberfl�che des Pellets\\
%hier Grafik 5b S.9 einf�gen
\begin{itemize}
\item[A:] Kat mit geringer Aktivit�t: gro�er $D_e$, kleiner $R$ (gro�er Wirkungsgrad, aber hoher Druckverlsut)
\item[B:] angestrebter Betriebszustand: hoher Wirkungsgrad bei gro�em $\Phi$
\item[C:] Kat mit hoher Aktivit�t: niedriger $D_e$, gro�er $R$ (Diffusion hemmt Reaktionsgeschwindigkeit)
\end{itemize}
Bei starker Diffusionshemmung findet die Reaktion nur in der Randzone statt. 
$\rightarrow$ schnellere Deaktivierung, da Kat unvollst�ndig genutzt\\
Einfluss der Reaktionsordnung:\\
{\it was ist hier wirklich wichtig?}
\subsection{Einfluss der Filmdiffusion auf die heterogene Katalyse}
Ammoniakoxidation:\\
\[ \quad \chemie{NH_3} \chemie{PLUS} \chemie{5O_2} \chemie{EQUILIBRIUM}{Pt-Netz}{ca. 900�C} \chemie{4NO} \chemie{PLUS} \chemie{6H_2O} \qquad \Delta H_R = - 906 \frac{kJ}{mol} \]
Kontaktzeit: $1/1000\,s$ (!)\\
laminare Filmdicke: $\delta=f\left(u,\rho,\nu,d_p\right)$\\
Filmtheorie (1. {\sc Fick}'sches Gesetz):\\
\[ \fbox{$ \displaystyle J=-D\frac{dc}{dx}=\underbrace{\frac{D}{\delta}}_{\beta}\cdot\left(c_b-c_s\right)$} \]
$\beta=f\left(u,\rho,\nu,d_p,D\right)$: Massentransportkoeffizient $\left[\frac{m}{s}\right]$\\
{\sc Biot}-Zahl:\\
\[ \fbox{$ \displaystyle Bi_m=\frac{\beta\cdot R}{D_e}=\frac{\mbox{Massentransport}}{\mbox{Porendiffusion}}$} \]
�bliche Gr��enordnung: $Bi_m=100...200$\\
f�r $Bi_m>100\rightarrow$ Filmdiffusion ist vernachl�ssigbar\\
\[ \eta=f\left(\underbrace{Bi_m}_{Filmdiffusion};\,\underbrace{\Phi}_{Porendiffusion}\right) \]
\subsection{Einfluss von W�rme- und Massentransport}
{\sc Prater}-Zahl:
\[ \fbox{$ \displaystyle \beta_{Pr}=\frac{c_b\cdot D_e\cdot\left(-\Delta H_R\right)}{\lambda_e\cdot T_b}=\frac{\mbox{max. $\Delta T$ im Pellet}}{\mbox{Bulk-Temperatur}}=\frac{\Delta T_{max}}{T_b}$} \]
$\beta_{Pr}>0$: exotherme Reaktion\\
$\beta_{Pr}<0$: endotherme Reaktion\\
%was hat es mit \eta>1 auf sich?
{\sc Arrhenius}-Zahl (Einfluss der Temperatur auf die Reaktion):\\
\[ \fbox{$ \displaystyle \gamma=\frac{E_A}{RT_b}$} \]
Keine Transporthemmung (willk�rliche Festlegung):\\
$\eta=\frac{r_{obs}}{r\left(T_b,\,c_b\right)}=1\pm 0,05$\\
{\sc Weisz-Prater}-Kriterium:\\
\[ \fbox{$ \displaystyle \Phi=\frac{r_{obs}\cdot R^2}{c_s\cdot D_e}$} \]
$\Phi$: {\sc Weisz}-Modul\\
$\Phi=\phi^2\cdot \eta$\\
Keine Limitierung durch Porendiffusion, wenn:\\
$\Phi<6$: Reaktion 0. Ordnung\\
$\Phi<1$: Reaktion 1. Ordnung\\
$\Phi<0,3$: Reaktion 2. Ordnung\\

{\sc Weisz-Hicks}-Kriterium:\\
$\rightarrow$ Pellet nicht isotherem, ausschlie�lich Porendiffusion\\
Keine Massen- und W�rmelimitierung falls $\Phi\cdot \exp{\left(\frac{\gamma\cdot\beta_{Pr}}{1+\beta_{Pr}}\right)}<1$

\subsection{Zweifilm-Theorie}
\subsubsection{Basis}
\begin{itemize}
\item Stagnierende Filme
\item Konzentrationsgradienten
\item Gleichgewicht an Phasengrenzfl�che ($\mu^G = \mu^L$)
\item Diffusion bewirkt Transport durch Grenzfl�che
\end{itemize}

\[ \fbox{$ \displaystyle J_{i,g} = - D_{i,g} \frac{\Delta c_{i,g}}{\delta_g} = J_{i,l} = - D_{i,l} \frac{\Delta c_{i,l}}{\delta_l} $} \]
Treibende Kraft: Chemisches Potential bzw. $\Delta c_i = c_i^* - c_i$!

Stoff�bergangskoeffizienten:
\[ K = \frac{D_i,P}{\delta_P} \]

{\sc Henry} Gesetz (verd�nnte L�sung Gas in Fl�ssigkeit:
\[ p_i^* = H_i c_i^* \]

{\sc Nernst}'sches Gesetz (Liquid/Liquid)
\[ c_{i,I}^* = K_{N} c_{i,II}^* \]

\subsubsection{Bilanz}
Acc = In - Out + React\\
station�r: Acc = 0 \\
\[ (J_1 a)_{y} + (J_1 a)_{y +dy} = k' c_1 c_2 a dy \]
\[ J_1 = -D_{1,l} \left.\frac{dc}{dy}\right|_{y} \qquad \left. \frac{dc_1}{dy} \right|_{y+dy} = \frac{dc_1}{dy} + \frac{dc^2}{dy}dy \]
\[ \rightarrow D_{1,l} \frac{d^2c_1}{dy^2} = k c_1 c_2 \qquad D_{2,l} = \frac{d^2c_2}{dy^2} = k c_1 c_2 \]
Vereinfachungen:
\[ p_1^* = p_{1,g} \qquad c_1^* = \frac{p_{1,g}}{H_1} \qquad k = k' c_2 \]
\[ y = 0: \qquad c_1=c_1^* \qquad c_2=c_{2,l} \]
\[ y = \delta_l: \qquad c_1 = c_{1,l} \qquad c_2 = c_{2,l} \]
\[ \fbox{$ \displaystyle d_{1,l} \frac{d^2c_2}{dy^2} = k c_1 $} \]
Mit der dimensionslosen Gr��e {\sc Ha}:
\[ Ha = \delta_l \sqrt{\frac{k}{D_{1,l}}} = \frac{\delta_l}{D_{1,l}} \sqrt{k D_{1,l}} = \frac{1}{k_{1,l}} \sqrt{k D_{1,l}} \]

\subsubsection{Chemische Reaktion und Stofftransport}
\begin{enumerate}
\item Chemische Reaktion im Bulk
\item Stofftransport durch Fl�ssigkeitsfilm ist langsamer als Reaktion $\rightarrow$ Reaktion im Film
\end{enumerate}


\section{St�chiometrie chemischer Reaktionen}
\subsection{Allgemeines}
$N$ Komponenten: $A_1...A_N$ chemische Spezies\\
Schl�sselkomponenten: Mol�nderungen m�ssen bekannt / messbar sein, um eine Aussage �ber die Mol�nderungen der anderen Komponenten zu bekommen.
Schl�sselreaktionen:
\[ \frac{dc_i}{dt} = \sum_{j=1}^{M} \nu_{ij} r_j \]
Anzahl der Mole eines Elements:
\[ b_h = \sum_{i=1}^N \beta_{hi} n_i \quad \sum_{i=1}^N \beta_{hi} \Delta n_i = 0\]

\subsection{Element-Spezies-Matrix}
Bsp. Methanolherstellung.
\begin{table}[H]
\begin{tabular}{|ccc|ccc|cccc|} \hline
\multicolumn{3}{|c|}{ } & \multicolumn{7}{|c|}{$N = 7$} \\ \hline
$h$ & Elem & $i \rightarrow$    & 1 & 2 & 3 & 4 & 5 & 6 & 7 \\
    &      & Spez $\rightarrow$ & $C$ & $CH_4$ & $H_2O$ & $H_2$ & $CO$ & $CO_2$ & $ C_2H_6$\\ \hline
1   & C    &                        & 1 & 1 & 0 & 0 & 1 & 1 & 2 \\
2   & H    &                        & 0 & 4 & 2 & 2 & 0 & 0 & 6 \\
3   & O    &			  & 0 & 0 & 1 & 0 & 1 & 2 & 0 \\ \hline
\multicolumn{3}{|l|}{$L = 3$} & \multicolumn{3}{|c|}{gebunden} & \multicolumn{4}{|c|}{frei} \\ \hline
\end{tabular}
\end{table}
Matrix B hat den Rang $R_{\beta}=3$. \\
Anzahl der Spezies meist gr��er als Anzahl der Elemente. Dann miest $R_{\beta} = L$.\\
\[ R = N - R_{\beta} \]
\begin{itemize}
\item $R$ Zahl der Schl�sselkomponenten - Freie Unbekannte
\item $R_{\beta}$ gebundene Unbekannte - werden berechnet
\item Anzahl Schl�sselreaktionen = Anzahl Schl�sselkomponenten
\end{itemize}
Das System kann von unten gel�st werden: $\Delta n_{H_2} = - \Delta n_{CO} - 2 \Delta n_{CO_2}$ etc.

\subsection{Ermittlung der Schl�sselreaktionen}
\subsubsection{�ber homogene L�sung}
Ermittle �ber Linearkombination der gebundenen Komponenten deren st�chiometrische Koeffizienten:
\begin{table}[H]
\begin{tabular}{|ccc|cccc|} \hline
$\nu_{CH_4}$ & $\nu_{H_2O}$ & $\nu_{H_2}$ & $\nu_{CO}$ & $\nu_{CO_2}$ & $\nu_C$ & $\nu_{C_2H6}$ \\ \hline
$\nu_{CH_4,1}$ & $\nu_{H_2O,1}$ & $\nu_{H_2,1}$ & $1$ & $0$ & $0$ & $0$ \\ 
$\nu_{CH_4,2}$ & $\nu_{H_2O,2}$ & $\nu_{H_2,2}$ & $0$ & $1$ & $0$ & $0$ \\ 
$\nu_{CH_4,3}$ & $\nu_{H_2O,3}$ & $\nu_{H_2,3}$ & $0$ & $0$ & $1$ & $0$ \\ 
$\nu_{CH_4,4}$ & $\nu_{H_2O,4}$ & $\nu_{H_2,4}$ & $0$ & $0$ & $0$ & $1$ \\ \hline
\end{tabular}
\end{table}
Eine spezielle L�sung w�re z.B. 
\[ \chemie{CH_4} + \chemie{H_2O} \chemie{EQUILIBRIUM} 3\chemie{H_2} + \chemie{CO} \]
weitere spezielle L�sungen ergeben die Matrix:
\begin{table}[H]
\begin{tabular}{|ccc|cccc|} \hline
$\nu_{CH_4}$ & $\nu_{H_2O}$ & $\nu_{H_2}$ & $\nu_{CO}$ & $\nu_{CO_2}$ & $\nu_C$ & $\nu_{C_2H6}$ \\ \hline
$-1$ & $-1$ & $3$ & $1$ & $0$ & $0$ & $0$ \\ 
$-1$ & $-2$ & $4$ & $0$ & $1$ & $0$ & $0$ \\ 
$-1$ & $0$ & $2$ & $0$ & $0$ & $1$ & $0$ \\ 
$-2$ & $0$ & $1$ & $0$ & $0$ & $0$ & $1$ \\ \hline
\end{tabular}
\end{table}
Diese Gleichungen stellen die Schl�sselreaktionen dar.

\subsubsection{Aus einem Satz bekannter Reaktionen}
In der Praxis ist oftmals bekannt, welche Teil-Reaktionen ablaufen.
Somit kann �ber den {\sc Gauss}-Algorithmus gel�st werden.
Beim o.g. Beispiel:
\begin{table}[H]
\begin{tabular}{|l|ccccccc|} \hline
$i \rightarrow$    & 1 & 2 & 3 & 4 & 5 & 6 & 7 \\
Spez $\rightarrow$ & $CH_4$ & $H_2O$ & $H_2$ & $CO$ & $CO_2$ & $C$ & $C_2H_6$ \\ \hline
1 & -1 & -1 & 3 & 1 & 0 & 0 & 0 \\
2 & 0 & -1 & 1 & -1 & 1 & 0 & 0 \\
3 & -1 &0 &2 &0 &0 &1 &0 \\
4 & 0 &-1 &1 &1 &0 &-1& 0 \\
5 & 0 &0 &0 &-2 &1 &1 &0 \\
6 & -2 &0& 1 &0 &0 &0 &1 \\ \hline
\end{tabular}
\end{table}
Wird zu:
\begin{table}[H]
\begin{tabular}{|l|ccccccc|} \hline
$i \rightarrow$    & 1 & 2 & 3 & 4 & 5 & 6 & 7 \\
Spez $\rightarrow$ & $CH_4$ & $H_2O$ & $H_2$ & $CO$ & $CO_2$ & $C$ & $C_2H_6$ \\ \hline
1 & -1 & -1 & 3 & 1 & 0 & 0 & 0 \\
2 & 0 & -1 & 1 & -1 & 1 & 0 & 0 \\
3 & 0 &0 &-3 &-4 &2 &0 &1 \\
4 & 0 &0 &0 &-2 &1 &1& 0 \\
5 & 0 &0 &0 &0 &0 &0 &0 \\
6 & 0 &0& 0 &0 &0 &0 &0 \\ \hline
\end{tabular}
\end{table}
Hier sind die Schl�sselreaktionen direkt ersichtlich.

\subsection{Beziehungen zwischen St�chiometrie und Reaktionskinetik}
Um sinnvolle Erkenntnisse aus den Berechnungen zu erhalten, welche den Reaktionsverlauf oder Reaktionsmechanismus beschreibt, ist eine Bewertung notwendig.

\begin{enumerate}
\item $M = R_{\nu} = N - R_{\beta}$ - Anzahl der wirklichen Reaktionen entspricht der Anzahl der Schl�sselreaktionen.\\
  Bsp. 3 Isomere, 1 Schl�sselreaktion
  \begin{itemize}
     \item Parallelreaktion ($A_1 \rightarrow A_2$ und $A_1 \rightarrow A_3$)
     \item Folgereaktion ($A_1 \rightarrow A_2 \rightarrow A_3$)
  \end{itemize}
  Entscheidung durch Konzentrationsmessung von $A_3$ an $t=0$. Wenn $\frac{dc_3}{dt}=0$, dann 2.
\item $M > R_{\nu} = N - R_{\beta}$ - Mehr ablaufende Reaktionen als Schl�sselreaktionen
\item $M < R_{\nu} = N - R_{\beta}$ - Eine oder mehrere Schl�sselreaktionen sind kinetisch unm�glich.\\
  Bsp. Rohrzuckerinversion.
\end{enumerate}

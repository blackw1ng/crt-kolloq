\section{St�chiometrie chemischer Reaktionen}
\subsection{Allgemeines}
$N$ Komponenten: $A_1...A_N$ chemische Spezies\\
Schl�sselkomponenten: Mol�nderungen m�ssen bekannt / messbar sein, um eine Aussage �ber die Mol�nderungen der anderen Komponenten zu bekommen.
Schl�sselreaktionen:
\[ \frac{dc_i}{dt} = \sum_{j=1}^{M} \nu_{ij} r_j \]
Anzahl der Mole eines Elements:
\[ b_h = \sum_{i=1}^N \beta_{hi} n_i \quad \sum_{i=1}^N \beta_{hi} \Delta n_i = 0\]

\subsection{Element-Spezies-Matrix}
Bsp. Methanolherstellung.
\begin{table}[h]
\begin{tabular}{|ccc|ccc|cccc|} \hline
\multicolumn{3}{|c|}{ } & \multicolumn{7}{|c|}{$N = 7$} \\ \hline
$h$ & Elem & $i \rightarrow$    & 1 & 2 & 3 & 4 & 5 & 6 & 7 \\
    &      & Spez $\rightarrow$ & $C$ & $CH_4$ & $H_2O$ & $H_2$ & $CO$ & $CO_2$ & $ C_2H_6$\\ \hline
1   & C    &                        & 1 & 1 & 0 & 0 & 1 & 1 & 2 \\
2   & H    &                        & 0 & 4 & 2 & 2 & 0 & 0 & 6 \\
3   & O    &			  & 0 & 0 & 1 & 0 & 1 & 2 & 0 \\ \hline
\multicolumn{3}{|l|}{$L = 3$} & \multicolumn{3}{|c|}{gebunden} & \multicolumn{4}{|c|}{frei} \\ \hline
\end{tabular}
\end{table}
Matrix B hat den Rang $R_{\beta}=3$. \\
Anzahl der Spezies meist gr��er als Anzahl der Elemente. Dann miest $R_{\beta} = L$.\\
\[ R = N - R_{\beta} \]
\begin{itemize}
\item $R$ Zahl der Schl�sselkomponenten - Freie Unbekannte
\item $R_{\beta}$ gebundene Unbekannte - werden berechnet
\item Anzahl Schl�sselreaktionen = Anzahl Schl�sselkomponenten
\end{itemize}
Das System kann von unten gel�st werden: $\Delta n_{H_2} = - \Delta n_{CO} - 2 \Delta n_{CO_2}$ etc.

\subsection{Ermittlung der Schl�sselreaktionen}



\section{Beschreibung einfacher Reaktionen}
\subsection{Potenzansatz - Power Law Expressions}
  \[ \fbox{$ \displaystyle \displaystyle  r = k c_1^{m_1} c_2^{m_2} = \frac{1}{\nu_i}\frac{dc_i}{dt} $} \]
  bzw. allgemeiner:
  \[ r = k \prod_i^N c_i^{m_i} \]

\subsection{Temperaturabh�ngigkeit einer Reaktion - {\sc Arrhenius}-Law}
  \begin{itemize}
  \item {\sc Arrhenius}-Gleichung 
  \[ k = k_0 \exp{-\frac{E}{RT}} \]
  \item im {\sc Arrhenius}-Plot: $\ln k$ gegen $\frac{1}{T}$ wegen
  \[ \ln k = - \frac{E}{R} \frac{1}{T} + \ln k_0 \]
  \item Aktivierungsenergie $E$ liegt normalerweise im Bereich $40...200\, \frac{kJ}{mol}$
  \item Sto�faktor $k_0 \left[\frac{mol^{1-m}}{s}\right]$
  \item Faustregel: Temperaturerh�hung um $10\,K$ verdoppelt Reaktionsgeschwindigkeit!
  \end{itemize}

\subsection{Definition der Reaktionsgeschwindigkeit}
  \begin{itemize}
  \item \[ r^* = \frac{1}{\nu_i} \frac{dn_i}{dt} = \frac{d\xi}{dt} \]
  Achtung: Extensive Gr��e!
  \item Homogene Reaktionen (Normierung auf Reaktionsvolumen bzw. -Masse)
  \[ \fbox{$ \displaystyle \displaystyle  \frac{r^*}{V} = \frac{1}{V}\frac{d\xi}{dt} $} \quad \left[\frac{mol}{m^3\,s}\right] \qquad \fbox{$ \displaystyle \displaystyle  \frac{r^*}{m} = \frac{1}{m}\frac{d\xi}{dt} $} \quad \left[\frac{mol}{kg\,s}\right] \]
  \item Heterogene Reaktionen (Normierung auf Katalysatoroberfl�che bzw. -Masse)
  \[ \fbox{$ \displaystyle \displaystyle  \frac{r^*}{A} = \frac{1}{A}\frac{d\xi}{dt} $} \quad \left[\frac{mol}{m^2\,s}\right] \qquad \fbox{$ \displaystyle \displaystyle  \frac{r^*}{m_{cat}} = \frac{1}{m_{cat}}\frac{d\xi}{dt} $} \quad \left[\frac{mol}{kg\,s}\right] \]
  \item Sonderfall: Reaktionsvolumen $V = const$
  \[ \fbox{$ \displaystyle \displaystyle  r = \frac{1}{\nu_i} \frac{dc_i}{dt} $} \]
  \item Achtung: Reaktionsgeschwindigkeit wird auf eine \emph{Reaktion} bezogen, die Bildungs- bzw. Verbrauchsgeschwindigkeit wird hingegen auf eine \emph{Spezies} bezogen!
  \end{itemize}

\subsection{Bildungs- bzw. Verbrauchsrate}
\[ \fbox{$ \displaystyle \displaystyle  R_i = \frac{dc_i}{dt} = \sum_{j=1}{M} \nu_{ij} r_i $} \]
bei $V = const$

\subsection{Reaktionen 1. Ordnung}
Umwandlung eines Molek�ls in ein anderes: \[ \chemie{A_1} \chemie{GIVES} \chemie{A_2}\]
\[ \fbox{$ \displaystyle \displaystyle  R = \frac{dc_1}{dt} = \nu_1 r = - k\cdot c_1 $} \]

\subsection{Reaktionen 2. Ordnung}
\subsubsection{2 Gleichartige Reaktanden}
\[ \chemie{2 A_1}  \chemie{GIVES} \chemie{A_2} \]
\[ \fbox{$ \displaystyle \displaystyle  R = \frac{dc_1}{dt} = - 2 r = - 2 k c_1^2 $} \]
\subsubsection{2 Ungleiche Reaktanden}
\[ \chemie{A_1} \chemie{PLUS} \chemie{A_2} \chemie{GIVES} \chemie{A_3}\]
\[ r = - \frac{dc_1}{dt} = - \frac{dc_2}{dt} = k c_1 c_2 \quad \leadsto \quad \fbox{$ \displaystyle \displaystyle  R = - k c_1 c_2 $} \]
Spezialfall: Reaktion Pseudo 1. Ordung $\leftrightarrow$ Ein Reaktand liegt in unendlicher Konzentration vor.\\
In diesem Fall: $c_2 = c_{20}$
\[ \fbox{$ \displaystyle \displaystyle  \frac{dc_1}{dt} = - \underbrace{k c_{20}}_{= k_{eff}} c_1 = - k_{eff} c_1 $} \quad \mbox{bei } c_{20} \gg c_{10} \]


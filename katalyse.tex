\section{Einf�hrung in die Katalyse}
\subsection{Grundlagen}
Ein Katalysator
\begin{itemize}
\item beschleunigt eine chemische Reaktion, ohne das thermodynamische Gleichgewicht der Reaktion zu verschieben.
\item er�ffnet einen neuen Reaktionsweg mit geringerer Aktivierungsenergie.
\item vermeidet stabile Zwischenprodukte durch den katalytischen Reaktionsweg.
\end{itemize}
Vorteile sind demnach:
\begin{itemize}
\item milde Reaktionsbedingungen
\item Kosteneffizienz
\item Umweltfreundlichkeit
\end{itemize}
Die G�te eines Katalysators wird durch folgende Merkmale bestimmt:
\begin{enumerate}
\item Selektivit�t
\item Aktivit�t
\item Lebensdauer
\end{enumerate}

\subsection{Katalytischer Kreislauf}
\begin{enumerate}
\item Aktivierung des Katalysators
\item Aktiver Katalysator bindet an Substrat I
\item Der Katalysator-Substrat-Komplex bindet an Substrat II
\item Es bildet sich ein �bergangszustand, welcher in einen Katalysator-Produkt-Komplex �bergeht
\item Produkt und Katalysator trennen sich und es geht bei (2) weiter.
\end{enumerate}

\subsection{Homogene und Heterogene Katalyse}

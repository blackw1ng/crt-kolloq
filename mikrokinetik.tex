\section{Mikrokinetik}
\begin{itemize}
\item Langmuir-Hinshelwood:
\item Elay-Rideal:
\end{itemize}
$\rightarrow$ Prinzip des Geschwindigkeit bestimmenden Schrittes (RDS)
Bsp.: Langmuir-Hinshelwood:
\begin{enumerate}
\item $|\nu_1|\chemie{A_1} \chemie{PLUS} |\nu_2|\chemie{A_2} \chemie{EQUILIBRIUM} |\nu_3|\chemie{A_3} \chemie{PLUS} |\nu_4|\chemie{A_4}$
mit $\nu_1=\nu_2=-1$; $\nu_2=\nu_4=1$
Annahme macht Sinn, da Oberfl�che meist bimolekular.
\begin{itemize}
\item jeder Reaktand ist an einem Zentrum adsorbiert; Reaktion im adsorbierten Zustand
\item jedes Produkt an einem Zentrum
\item Anzahl der aktiven Zentren = konstant �ber die Zeit
\item alle Reaktionen werden als Gleichgewichtsreaktionen behandelt
\end{itemize}
Ablauf
\begin{itemize}
\item Sorption der Reaktanden
$A_i+l_0 \chemie{EQUILIBRIUM}{k_{ads,i}}{k_{des,i}} A_il$ mit $i=1,2$
\item Oberfl�chenreaktion
$A_1l+A_2l \chemie{EQUILIBRIUM}{k_1}{k_2} A_3l+A_4l$
\item Sorption der Produkte 
$A_i \chemie{EQUILIBRIUM}{k_{des,i}}{k_{ads,i}} A_i +l_0$ mit $i=3,4$
\end{itemize}
\item Geschwindigkeitsgleichungen f�r die Elementarreaktionen
\end{enumerate}


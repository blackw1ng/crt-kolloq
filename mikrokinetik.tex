\section{Mikrokinetik}
\begin{itemize}
\item Langmuir-Hinshelwood:
\item Elay-Rideal:
\end{itemize}
$\rightarrow$ Prinzip des Geschwindigkeit bestimmenden Schrittes (RDS)
Bsp.: Langmuir-Hinshelwood:

$|\nu_1|\chemie{A_1} \chemie{PLUS} |\nu_2|\chemie{A_2} \chemie{EQUILIBRIUM} |\nu_3|\chemie{A_3} \chemie{PLUS} |\nu_4|\chemie{A_4}$
mit $\nu_1=\nu_2=-1$; $\nu_2=\nu_4=1$
Annahme macht Sinn, da Oberfl�che meist bimolekular.
\begin{itemize}
\item jeder Reaktand ist an einem Zentrum adsorbiert; Reaktion im adsorbierten Zustand
\item jedes Produkt an einem Zentrum
\item Anzahl der aktiven Zentren = konstant �ber die Zeit
\item alle Reaktionen werden als Gleichgewichtsreaktionen behandelt
\end{itemize}
Ablauf
\begin{itemize}
\item Sorption der Reaktanden
$A_i+l_0 \chemie{EQUILIBRIUM}{k_{ads,i}}{k_{des,i}} A_il$ mit $i=1,2$
\item Oberfl�chenreaktion
$A_1l+A_2l \chemie{EQUILIBRIUM}{k_1}{k_2} A_3l+A_4l$
\item Sorption der Produkte 
$A_i \chemie{EQUILIBRIUM}{k_{des,i}}{k_{ads,i}} A_i +l_0$ mit $i=3,4$
\end{itemize}
Geschwindigkeitsgleichungen f�r die Elementarreaktionen
\begin{itemize}
\item Adsorption: $r_{ads,i}=k_{ads,i}p_1\Theta_0$ mit $i=1,2,3,4$
\item Desorption: $r_{des,i}=k_{des,i}\Theta_i$ mit $i=1,2,3,4$
\item Hinreaktion: $r_1=k_1\Theta_1\Theta_2$
\item R�ckreaktion: $r_2=k_2\Theta_3\Theta_4$
\end{itemize}
$\sum{i=1}{N}\Theta_i+\Theta_0=1$
Entwicklung einer Geschwindigkeitsgleichung auf Basis des RDS
$\rightarrow$ nur die Geschwindigeitsgleichung des RDS wird formuliert
$\rightarrow$ die unbekannten Belegungsgrade werden durch die Annahme bestimmt, dass alle anderen Teilschritte im Gleichgewicht sind.
Drei F�lle werden unterschieden:
\begin{itemize}
\item Oberfl�chenreaktion ist RDS
$r=k_1\Theta_1\Theta_2-k_2\Theta_3\Theta_4$
mit $r_{Sorption,Reaktand}=r_{Sorption,Priodukt}=0$
\item Sorption von Reaktand $A_1$ ist RDS
$r=k_{ads,1}p_1\Theta_0-k_{des_1}\Theta_1$
\item Sorption von Produkt $A_3$ ist RDS 
$r=k_{ads,3}p_3\Theta_0-k_{des_3}\Theta_3$
\end{itemize}
Allgemeine Sorptionsisotherme f�r Langmuir-Hinshelwood-Mechanismus:
  \[ \fbox{$ \Theta_i=\frac{K_ip_i}{1+\sum_{q=1}^{N}K_qp_q}\qquad$ mit $\qquad K_i=\frac{k_{ads,i}}{k_{des,i}}$ } \] 



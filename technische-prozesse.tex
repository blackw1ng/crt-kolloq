\section{Wichtige Gro�techn. Prozesse}
\subsection{Schwefels�ureherstellung}
\begin{enumerate}
\item Verbrennung von Schwefel zu $SO_2$
\item Oxidation von $SO_2$ $\quad \chemie{SO_2} \chemie{EQUILIBRIUM}{V_2O_5}{400..600�C} \chemie{SO_3} \qquad \Delta H_R = - 99 \frac{kJ}{mol} $
\item $SO_3$-Absorption an Wasser $\quad \chemie{SO_3} \chemie{PLUS} \chemie{H_2O} \chemie{GIVES} \chemie{H_2SO_4} \qquad \Delta H_R = - 132,5 \frac{kJ}{mol} $
\end{enumerate}

\subsection{Ammoniakherstellung}
Der aufw�ndigste Teil ist hierbei die bereitstellung des Synthesegases!
Dieses wird in mehreren Schritten ausgehend von Methan hergestellt:
\begin{enumerate}
\item Steam-Reforming
\[ \chemie{CH_4} + \chemie{H_2O} \chemie{EQUILIBRIUM} \chemie{CO} + \chemie{3 H_2} \qquad \Delta H_R = 206 \frac{kJ}{mol} \]
Stark endotherm. Durchf�hrung in Radiation-Convection-�fen. Ni-Katalysiert.
\item Luftzugabe - Autothermal-Reforming:
\[ \chemie{H_2} + \chemie{O_2}_{Luft} \chemie{EQUILIBRIUM} \chemie{H_2O} \]
Wird gemacht um das $N_2$ dazu zu bekommen.
\item Wasser-Gas-Shift
\[ \chemie{CO} + \chemie{H_2O} \chemie{EQUILIBRIUM} \chemie{CO_2} + \chemie{H_2} \qquad \Delta H_R = -41 \frac{kJ}{mol} \]
Wichtig: Genaue Wasserzugabe und niedrige Temperatur. Praktisch: 2 Reaktoren (High+Low Temp=
\item Entfernung von $CO_2$ durch {\sc Rectisol}-Absorptionsverfahren
\item Methanisierung
\[ \chemie{CO} + \chemie{3 H_2} \chemie{EQUILIBRIUM} \chemie{CH_4} + \chemie{H_2O} \qquad \Delta H_R = - 206 \frac{kJ}{mol} \]
Hoch Exotherm. 620K bei 30bar. Macht man um das Kat-Gift. $CO$ in ppm-Bereich abzusenken.
\item Ammoniaksynthese
\[ \chemie{N_2} \chemie{PLUS} \chemie{H_2} \chemie{EQUILIBRIUM} \chemie{2 NH_3} \qquad \Delta H_R = - 91 \frac{kJ}{mol} \]
Optimierungsproblem zwischen Temperatur (hoch: wenig $NH_3$, niedrig: zu langsam) und Druck ($100-300 bar$).
Wegen Exothermie: Zwischenk�hlung! 
\end{enumerate}

\subsection{Methanolherstellung}
\[ \chemie{CO} \chemie{PLUS} \chemie{H_2} \chemie{EQUILIBRIUM} \chemie{CH_3OH} \qquad \Delta H_R = - 92 \frac{kJ}{mol} \]


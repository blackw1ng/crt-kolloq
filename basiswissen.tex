\section{Basiswissen}
\subsection{Konventionen}
\begin{itemize}
\item Index $h = 1...L$ - Elemente
\item Index $i = 1...N$ - Komponenten / Spezies
\item Index $j = 1...M$ - Teilreaktionen
\item Exponent $m_i$ - Teilreaktionsordnung bezogen auf $A_i$ (St�ch. Koeffizient?!)
\item St�chimetrische Koeffizienten $\nu_i$; $\nu_i < 0$: Edukt, $\nu_i > 0$: Produkt
\item Elementkoeffient des Elements $h$ in Spezies $i$: $\beta_{hi}$
\item Geschwindigkeitskoeffizient $k$ $\left[ \frac{mol}{l \cdot s}\right]$
\item Bildungs- bzw. Verbrauchsgeschwindigkeit $R$ \emph{ Rate of formation / deplition}
\item Reaktionsgeschwindigkeit $r$
\end{itemize}

\subsection{Definitionen}
\begin{itemize}
\item \emph{Extensive} Terme: Variablen eines Systems, welche sich bei Teilung des Systems in zwei Teilsysteme �ndern. Bsp: $V$, $m$, $n$
\item \emph{Intensive} Terme: Variablen eines Systems, die bei Teilung des Systems in Teilsysteme konstant bleiben. Bsp: $T$, $p$, $\rho$
\item Extensive Variablen k�nnen durch Normierung �ber $m$ oder $V$ zu \emph{Pseudo-Intensiven} Variablen umgewandelt werden.
\item Massenerhaltung 
\[ \sum_{i=0}^{N} \nu_i M_i = 0 \]
\end{itemize}

\subsubsection{Reaktionslaufzahl}
  \begin{displaymath}
  \fbox{$ \displaystyle \xi = \frac{n_i - n_{i0}}{\nu_i} $}
  \end{displaymath}
  \begin{itemize}
  \item $\xi = 0$: Reaktionsbeginn
  \item $c_i = c_{i0} + \nu_i \lambda$
  \item $m_i = m_{i0} + M_i \nu_i \xi$
  \item $n_i = n_{i0} + \nu_i \xi$
  \end{itemize}

\subsubsection{Umsatz}
  \[ \fbox{$ \displaystyle \mbox{Umsatz} = \frac{\mbox{Anteil an bereits Reagierter Komponente}}{\mbox{Anfangsmenge der Komponente}}$} \]
  \[ \fbox{$ \displaystyle X_i = \frac{\dot{n}_{i0} - \dot{n}_i}{\dot{n}_{i0}} $} \]

\subsubsection{Ausbeute}
  \[ \fbox{$ \displaystyle   \displaystyle   \displaystyle    \mbox{Ausbeute} = \frac{\mbox{Menge an gebildetem Produkt k}}{\mbox{Menge an zugegebener, limitierender Komponente i}} $} \]
  \[ \fbox{$ \displaystyle Y_{ki} = \frac{\dot{n}_k - \dot{n}_{k0}}{\dot{n}_{i0}} \frac{\left|\nu_i\right|}{\left|\nu_k\right|}$} \]

\subsubsection{Selektivit�t}
  \[ \fbox{$ \displaystyle   \displaystyle   \displaystyle    \mbox{Selektivit�t} = \frac{\mbox{Menge an gebildetem Produkt k}}{\mbox{Menge an umgesetztem Reaktanden i}} $} \]
  \[ \fbox{$ \displaystyle S_{ki} = \frac{\dot{n}_k - \dot{n}_{k0}}{\dot{n}_{i0} - \dot{n}_i} \frac{\left|\nu_i\right|}{\left|\nu_k\right|} = \frac{Y_{ki}}{X_i} $} \]

\subsubsection{Reaktorkenngr��en}
\begin{itemize}
\item Produktivit�t
\[ L = \dot{n}_k \]
\item Querschnittsbelastung
\[ G = \frac{m}{A t} \]
\item Raum-Zeit-Ausbeute
\[ STY = \frac{L}{V} = \frac{\dot{n}_k}{V} \]
\end{itemize}


\section{Grundlagen der Adsorption}
Bei der Adsorption unterscheidet man zwischen 
\begin{itemize}
\item Physisorption 
\item Chemisorption
\end{itemize}

\subsection{Eigenschaften}
\begin{table}[H]
\begin{tabular}{|l|c|c|} \hline
		& \bf Physisorption	& \bf Chemisorption \\ \hline
Adsorbens (s)	& Alle			& Wenige \\
Adsorpt (g)	& Alle			& Reaktive Gase \\
W�rme $\frac{kJ}{mol}$ & 8-25		& 20-420 \\
Geschwindigkeit	& sehr schnell		& variabel \\
$E_{ads}$	& < 4			& akt: 84, non akt: <4 \\
Temperatur	& niedrig		& hoch \\
Belegung	& auch Mehrlagig	& Einlagig \\
Reversibilit�t	& hoch			& meist niedrig \\ \hline
\end{tabular}
\end{table}

\subsection{Beschreibung}

{\sc Langmuir}-Adsorptions-Isotherme beschreibt den Belegungsgrad $\theta$.\\
Annahmen hierbei: Homogene Oberfl�che, keine Interaktion zwischen adsorbierten Molek�len.
\[ \fbox{$ \displaystyle  \theta = \frac{K p}{1 + K p} $} \]

Grenzf�lle: $K p \ll 1 \rightarrow \theta \approx K p$ \\
$K p \gg 1 \rightarrow \theta \approx 1$ \\

K kann �ber die {\sc Van't Hoff}'sche Reaktionsisochore beschrieben werden
\[ K = K_{\infty} \exp \left( \frac{- \Delta H_{ads}}{RT} \right) \]

Der Gesamtbelegungsgrad ergibt sich aus der Summe der Einzelbelegungen
\[ \theta = \sum_{i=1}^{N} \theta_i \]

\subsection{Sorptionsisothermen}
\begin{table}[H]
\begin{tabular}{|l|c|c|} \hline
\bf Autor	& \bf Isotherme		& \bf Anwendung \\ \hline
\sc Langmuir	& $\nu / \nu_m = Kp / (1+Kp)$ & P + C \\
\sc Henry	& $\nu = a p$	& P + C \\ 
\sc Freundlich	& $\nu = k p^{1/n}$ & P + C \\
\sc Slygin-Fumkin & $\nu / \nu_m = k_1 \ln (k_2 p)$ & C \\
\sc BET		& $\frac{p}{\nu(p_0-p)} = \frac{1}{\nu_mb}+\frac{b-1}{\nu_mb}\frac{p}{p_0}$ & P (multilayer) \\ \hline
\end{tabular}
\end{table}

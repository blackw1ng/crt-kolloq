\section{Auswertung kinetischer Daten}
\subsection{Scale-Up}
Problem beim Scale-Up von der Laboranlage zur gro�technischen Anlage sind:
\begin{itemize}
\item Form des Reaktors
\item W�rmezu- und abfuhr
\item Str�mungsbedingungen
\item Vermischungsverhalten
\end{itemize}
Als Methoden f�r den Scale-Up ergeben sich:
\begin{table}[H]
\begin{tabular}{c|c} \hline
\bf klassisch	& \bf modern \\ \hline
emprische Betrachtung & Detailverst�ndnis + Modellbildung \\
stufenweiser Scale-U & in einem Schritt \\
kostenintensiv & interdisziplin�r \\
Grundoperationen & Prozessdenken \\
\end{tabular}
\end{table}

\subsection{Ziel kinetischer Messungen}
Mikrokinetik
\begin{itemize}
\item nicht Transporlimitiert
\item Beschreibung durch intrinsische Kinetik oder Kenntnis des Mechanismus
\item Formalkinetisch vereinfachende Annahmen
\end{itemize}
Makrokinetik
\begin{itemize}
\item Effektivkinetik mit Transporteinfluss
\item nicht getrennt von Transport
\item Beschreibung durch Scale-down 
\end{itemize}

\subsection{Prinzipien von Betriebsweise und Bauart}
Ist der Reaktor
\begin{itemize}
\item komplex oder einfach?
\item isotherm, adiabat oder polytrop?
\item Homogen oder Heterogen?
\end{itemize}
L�sung: Bestimmung von Konzentrationen in Abh�ngigkeit von der Zeit ($c= f(\tau)$ bzw $c = f(t_R)$)!\\
BSTR: Zeitkonstante = $t_R$\\
PFTR: Zeitkonstante = $\tau$

\subsubsection{Differentialreaktoren}
Bei Differentialreaktoren kan bei kleinen Ums�tzen die Reaktionsgeschwindigkeit direkt bestimmt werden:
\[ \nu_{i} r = \frac{c_{i0}-(c_{i0}-dc_i)}{dt} = - c_{i0} \frac{dX_i}{dt} \]
In der Praxis ist allerdings $\frac{dX_i}{dt}$ schwer bestimmtbar. Damit ist das Ergebnis einem $T$ bzw. $c$ nicht mehr genau zuzuordnen!

\subsubsection{Schlaufenreaktor}
Ein idealer CSTR verh�lt sich wie ein Schlaufenreaktor - Also gro�er Recycle Strom, gradientenlos.\\
Die Stoffmengen�nderungsgeschwindigkeit ergibt sich zu:
\[ R_i = - \frac{\dot{n}_{i0} - \dot{n}_i}{V} \qquad R_k = \frac{N_{k0}dX_k}{m_{cat}} \]

\subsection{Beispiele f�r spezielle Rohrreaktoren}
\subsubsection{Heterogen Katalysierte Reaktionen}
Hier w�hlt man meist
\begin{itemize}
\item Rohrreaktor mit F�llung $\rightarrow$ Festbettreaktor
\item Schlaufenreaktor mit innerem und �u�erem Kreislauf
\end{itemize}
\[ r = \frac{1}{\nu_i} \frac{dn}{dt} \frac{1}{m_{kat}} \]

\subsubsection{Fluid-Fluid-Reaktoren}
Hier ist der Stofftransport von sehr gro�er Bedeutung!
\begin{enumerate}
\item Mikrokinetik oder Transportlimitierung bestimmen
\item Bestimmung der Makrokinetik in Reaktor mit bekannter Fluiddynamik und Austauschfl�che
\end{enumerate}


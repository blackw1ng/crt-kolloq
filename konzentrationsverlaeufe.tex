\section{Konzentrationsverl�ufe spezieller Reaktionen}
\subsection{Reversible Reaktion 1. Ordnung}
\[ \chemie{A_1} \chemie{EQUILIBRIUM}{k1}{k2} \chemie{A_2} \]
\[ R_1 = \frac{dc_1}{dt} = - k_1 c_1 + k_2 c_2 \]
\[ K = \frac{c_2^*}{c_1^*} = \frac{c_{10} - c_1^*}{c_1^*} \]
$^*$ bedeutet: Gleichgewichtskonzentration
\[ \ln \frac{c_1 - c_1^*}{c_{10} - c_1^*} = - (k_1+k_2)t \]
{\bf Hier Bild  Konzentrationsverlauf}

\subsection{Parallelreaktion}
\[ \chemie{A_1} \chemie{GIVES}{k_1} \chemie{A_2} \hbox{gleichzeitig}  \chemie{A_1} \chemie{GIVES}{k_2} \chemie{A_3} \]
\[ \frac{dc_1}{dt} = - (k_1 + k_2) c_1 \quad \leadsto \quad \frac{c_1}{c_{10}} = \exp\left[-(k_1+k_2)t\right] \]
{\bf Hier Bild  Konzentrationsverlauf}
Hier auch wichtig: Globale Selektivit�t \[ S = \frac{c_2 - c_{20}}{c_3 - c_{30}} = \frac{k_1}{k_2} \]

\subsection{Folgereaktionen}
\[\chemie{A_1} \chemie{GIVES}{k_1} \chemie{A_2} \chemie{GIVES}{k_2} \chemie{A_3}\]

\subsubsection{Bildungsgeschwindigkeiten}
\[ R_1 = \frac{dc_1}{dt} = -k_1 c_1 \quad \leadsto \quad c_1(t) = c_{10} \exp(-k_1t) \]
\[ R_2 = \frac{dc_2}{dt} = k_1c_1 - k_2c_2 \]
\[ R_3 = \frac{dc_3}{dt} = k_2 c_2 \]
\subsubsection{Quastistationarit�tsprinzip nach {\sc Bodenstein}}
Wenn $\frac{dc_2}{dt}$ sehr klein (wegen $k_2 \gg k_1$), dann ist die Konzentration der Zwischenprodukte (nach einer Induktionszeit) \emph{quasistation�r}.
Daraus folgt: \fbox{$ \displaystyle \frac{dc_2}{dt} \approx 0$}
\subsubsection{Konzentrationsverl�ufe}
\begin{enumerate}
\item $k_1 = k_2$
\item $k_1 = 20 k_2 \qquad k_1 \gg k_2$
\item $20 k_1 = k_2 \qquad k_1 \ll k_2$
\end{enumerate}
\subsubsection{Anwendung des {\sc Bodenstein}-Prinzips}
\[ \frac{dc_2}{dt} = k_1c_1 - k_2 c_2 \approx 0 \quad \leadsto k_1 c_1 = k_2 c_2 \]
mit
\[ c_1 = c_{10} \exp\left(-k_1t\right) \quad \leadsto \quad c_2 = \frac{k_2}{k_1} c_{10} \exp\left(k_1\right) \]


\section{Konzentrationsverl�ufe spezieller Reaktionen}
\subsection{Reversible Reaktion 1. Ordnung}
\[ \chemie{A_1} \chemie{EQUILIBRIUM}{k1}{k2} \chemie{A_2} \]
\[ R_1 = \frac{dc_1}{dt} = - k_1 c_1 + k_2 c_2 \]
\[ K = \frac{c_2^*}{c_1^*} = \frac{c_{10} - c_1^*}{c_1^*} \]
$^*$ bedeutet: Gleichgewichtskonzentration
\[ \ln \frac{c_1 - c_1^*}{c_{10} - c_1^*} = - (k_1+k_2)t \]
\begin{figure}[H]
\beginpicture
\setcoordinatesystem units <2.5cm,2.5cm>
\setplotarea x from 0 to 2, y from 0 to 1
 
\linethickness 1pt
\setquadratic
\setsolid
\plot 0 1 0.5 0.5 2 0.25 /
\setdots <1pt>
\plot 0 0 0.5 0.5 2 0.75 /
\endpicture
\end{figure}

\subsection{Parallelreaktion}
\[ \chemie{A_1} \chemie{GIVES}{k_1} \chemie{A_2} \quad \hbox{gleichzeitig} \quad \chemie{A_1} \chemie{GIVES}{k_2} \chemie{A_3} \]
\[ \frac{dc_1}{dt} = - (k_1 + k_2) c_1 \quad \leadsto \quad \frac{c_1}{c_{10}} = \exp\left[-(k_1+k_2)t\right] \]
\begin{figure}[H]
\beginpicture
\setcoordinatesystem units <2.5cm,2.5cm>
\setplotarea x from 0 to 2, y from 0 to 1
\axis bottom /
\axis left ticks unlabeled at 0 1 / /
\axis right /
\put {$t$} at 2.05 -0.02
\put {$c_{1,0}$} at -0.15 1
\put {$c_1$} at 2.12 0.1
\put {$c_2$} at 2.12 0.6
\put {$c_3$} at 2.12 0.35
\linethickness 1pt
\setquadratic
\setsolid
\plot 0 1 0.8 0.3 2 0.1 /
\setdots <1pt>
\plot 0 0 0.8 0.4 2 0.6 /
\setdashes <2pt>
\plot 0 0 0.8 0.2 2 0.35 /
\endpicture
\end{figure}
Hier auch wichtig: Globale Selektivit�t \[ S = \frac{c_2 - c_{20}}{c_3 - c_{30}} = \frac{k_1}{k_2} \]

\subsection{Folgereaktionen}
\[\chemie{A_1} \chemie{GIVES}{k_1} \chemie{A_2} \chemie{GIVES}{k_2} \chemie{A_3}\]

\subsubsection{Bildungsgeschwindigkeiten}
\[ R_1 = \frac{dc_1}{dt} = -k_1 c_1 \quad \leadsto \quad c_1(t) = c_{10} \exp(-k_1t) \]
\[ R_2 = \frac{dc_2}{dt} = k_1c_1 - k_2c_2 \]
\[ R_3 = \frac{dc_3}{dt} = k_2 c_2 \]
\subsubsection{Quastistationarit�tsprinzip nach {\sc Bodenstein}}
Wenn $\frac{dc_2}{dt}$ sehr klein (wegen $k_2 \gg k_1$), dann ist die Konzentration der Zwischenprodukte (nach einer Induktionszeit) \emph{quasistation�r}.
Daraus folgt: \fbox{$ \displaystyle \frac{dc_2}{dt} \approx 0$}
\subsubsection{Konzentrationsverl�ufe}
\begin{enumerate}
\item $k_1 = k_2$
\begin{figure}[H]
\beginpicture
\setcoordinatesystem units <2.5cm,2.5cm>
\setplotarea x from 0 to 2, y from 0 to 1
\axis bottom ticks numbered from 0 to 2 by 0.5 /
\axis left ticks numbered from 0 to 1 by 0.5 /
\axis top /
\axis right /
\put {$\frac{t}{t_{0,5}}$} at 2.2 -0.1
\linethickness 0.1pt
\setdots <2pt>
\putrule from 0 0.5 to 2 0.5
\putrule from 0.5 0 to 0.5 1
\putrule from 1 0 to 1 1
\putrule from 1.5 0 to 1.5 1
\putrule from 2 0 to 2 1
\linethickness 1pt
\setquadratic
\setsolid
\plot 0 1.00 0.5 0.45 2 0.02 /
\setdots <1pt>
\plot 0 0 0.6 0.4 2 0.15 /
\setdashes <2pt>
\plot 0 0 1 0.5 2 0.8 /
\endpicture
\end{figure}
\item $k_1 = 20 k_2 \qquad k_1 \gg k_2$
\begin{figure}[H]
\beginpicture
\setcoordinatesystem units <2.5cm,2.5cm>
\setplotarea x from 0 to 2, y from 0 to 1
\axis bottom ticks numbered from 0 to 2 by 0.5 /
\axis left ticks numbered from 0 to 1 by 0.5 /
\axis top /
\axis right /
\put {$\frac{t}{t_{0,5}}$} at 2.2 -0.1
\linethickness 0.1pt
\setdots <2pt>
\putrule from 0 0.5 to 2 0.5
\putrule from 0.5 0 to 0.5 1
\putrule from 1 0 to 1 1
\putrule from 1.5 0 to 1.5 1
\putrule from 2 0 to 2 1
\linethickness 1pt
\setquadratic
\setsolid
\plot 0 1.00 0.2 0.25 0.5 0 /
\setdots <1pt>
\plot 0 0 0.08 0.5 0.25 0.75 /
\plot 0.25 0.75 0.3 0.77 0.4 0.75 /
\plot 0.4 0.75 1 0.5 2 0.25 /
\setdashes <2pt>
\plot 0.1 0  1 0.5  2 0.75 /
\endpicture
\end{figure}

\item $20 k_1 = k_2 \qquad k_1 \ll k_2$
\begin{figure}[H]
\beginpicture
\setcoordinatesystem units <2.5cm,2.5cm>
\setplotarea x from 0 to 2, y from 0 to 1
\axis bottom ticks numbered from 0 to 2 by 0.5 /
\axis left ticks numbered from 0 to 1 by 0.5 /
\axis top /
\axis right /
\put {$\frac{t}{t_{0,5}}$} at 2.2 -0.1
\linethickness 0.1pt
\setdots <2pt>
\putrule from 0 0.5 to 2 0.5
\putrule from 0.5 0 to 0.5 1
\putrule from 1 0 to 1 1
\putrule from 1.5 0 to 1.5 1
\putrule from 2 0 to 2 1

\linethickness 1pt
\setquadratic
\setsolid
\plot 0 1.00 0.5 0.45 2 0.02 /
\setdots <1pt>
\setlinear
\plot 0 0  0.1 0.05  1 0.03  1.5 0.02  2 0.02 /
\setquadratic
\setdashes <2pt>
\plot 0 0 1 0.5 2 0.8 /
\endpicture
\end{figure}

\end{enumerate}
\subsubsection{Anwendung des {\sc Bodenstein}-Prinzips}
\[ \frac{dc_2}{dt} = k_1c_1 - k_2 c_2 \approx 0 \quad \leadsto k_1 c_1 = k_2 c_2 \]
mit
\[ c_1 = c_{10} \exp\left(-k_1t\right) \quad \leadsto \quad c_2 = \frac{k_2}{k_1} c_{10} \exp\left(k_1\right) \]

